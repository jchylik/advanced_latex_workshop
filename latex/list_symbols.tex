% ========================
% list  of symbols
%=======================
\begin{onehalfspace}
\begin{tabular}{l l p{0.7\textwidth} }
\hline
		notation  & unit & meaning \\
\hline \\
		\( \sim \) & \( \cdot \) & similar - assignment of probability distribution\\	
		\( \propto \) & \( \cdot \) & proportional equivalent to; i.e.  equivalent up to a constant\\
		\( \overline{(\; \cdot \;)} \) & \( \cdot \) & horizontal averaging \\
		\( \overline{\varphi} \) & 	\( \cdot  \) & mean value of a quantity \( \varphi \)  \\
% 		\( \overline{\varphi}^{(l,k)} \) & 	\( \cdot  \) & mean value of a quantity \( \varphi \) over a subdomain \\		
% 		\( \varphi^{\prime}\) & 	\( \cdot  \) & variant part of a quantity \( \varphi \)  \\			
% 		\( \overline{(\; \cdot \;)}_{u}^{(u)} \) & \( \cdot \) &  averaging over the area of strong updraughts \\
% 		\( {(\; \cdot \;)}_{RE}\) & \(\ \cdot \) &  resolved values of a term in parenthesis\\
% 		\( \overline{(\; \cdot \;)}_{SG}\) & \(\ \cdot \) &  statistics of modelled subgrid values of a term in parenthesis\\	
% 		\( {(\; \varphi \;)}_{(\mathrm{sm}),\lambda}  \) & \( \cdot \) & smoothing of a series of variable \( \varphi \)  over smoothing length \(\lambda \) \\	
% 		
% 		\( \triangle_{\varphi}\) & \( \cdot \)  & perturbation in a quantity \( \varphi \) \\
% 		\\
% 		
% 		\( a_u \) & \( \mathrm{m} \mathrm{s}^{-1}  \) & fraction of the area taken by strong updraughts \\	
% 		
% 		\( C_{p} \)  & \(\mathrm{J} \; \mathrm{kg}^{-1} \, \mathrm{K}^{-1}  \) & specific heat capacity at constant pressure (isobaric mass heat capacity)\\
% 		
% 		\( d_{\mathrm{(h)}} \) & \( \mathrm{m} \) & length of a side of block in a heterogeneity pattern \\			
% 		\( E_{k} \)  & \(\mathrm{J} \; \mathrm{kg}^{-1}  \) & kinetic energy per unit of mass\\
% 		\( E_{w} \)  & \(\mathrm{J} \; \mathrm{kg}^{-1}  \) & kinetic energy of vertical motion\\
% 		\( h^{(\varphi)}_i  \) & \( \cdot \) & scalar flux of the quantity \( \varphi \) \\
% 		\( L_{e} \)  & \(\mathrm{J} \; \mathrm{kg}^{-1} \) & latent heat of evaporation \\
% 		
% 		
% 		\( N \)  & \( \scriptstyle{1} \) & number of gridpoints / number of measurements in a timeseries  \\
% 		\( N_{x} \)  & \( \scriptstyle{1} \) & number of gridpoints in the direction of the axis-\(x\)\\
% 		\( \mathsf{N}(\mu,\sigma^{2}) \)  & \( \cdot \) & normal distribution with mean \( \mu \) and standard deviation \(\sigma \)  \\
% 
% 		\( r \)  & \( \mathrm{kg} \; \mathrm{kg}^{-1} \) & water vapour mixing ratio \\	
% 		%-> do we use this one ?
% 		\( r_{l} \)  & \( \mathrm{kg} \; \mathrm{kg}^{-1} \)  & liquid water mixing ratio\\
% 		\( r_{\varphi} \)  & \( \cdot \)  & residua in a series of the variable \( \varphi \) \\
% 		
% 		\( q_{v} \)  & \( \mathrm{kg} \; \mathrm{kg}^{-1} \) & specific humidity \\
% 		\( q_{cl} \)  & \( \mathrm{kg} \; \mathrm{kg}^{-1}  \)  & cloud total water content \\
% 		\( q_{i} \)  & \( \mathrm{kg} \; \mathrm{kg}^{-1}  \)  & cloud ice water content \\
% 		\( q_{l} \)  & \( \mathrm{kg} \; \mathrm{kg}^{-1}  \)  & cloud liquid water content \\
% 		\( q_{t} \)  & \( \mathrm{kg} \; \mathrm{kg}^{-1}  \)  & total water content (total humidity) \\
% 
% 		\( q_{tr} \)  & \( \; \mathrm{kg}^{-1}  \)  & content of a passive aerosol tracer\\	
% 		
% 		\( Q_{LH} \)  & \( \mathrm{W} \, \mathrm{m}^{-2} \) & latent heat flux \\
% 		\( Q_{SH} \)  & \( \mathrm{W} \, \mathrm{m}^{-2} \)  & sensible heat flux \\
% 		
% 		\( P( A ) \)  & \( \cdot \)  & probability of an event A \\
% 		
% 		\\ \hline
	\end{tabular}
    \end{onehalfspace}

%   \begin{onehalfspace}
%   \begin{tabular}{l l p{0.7\textwidth} }
%   \hline
% 		notation  & unit & meaning \\
% 		\hline \\
% 		\( \mathbb{S}_{i,j} \) & \( \mathrm{s}^{-1} \) & rate of strain tensor \\
% 		\( S \) & \( \mathrm{s}^{-1} \) & modulos of the rate of strain tensor \\
% 		\( S_{\varphi,\alpha} \) & \( \cdot \) &  sample quantile of values of variable \( \varphi \) for probability \(alpha \)\\
% 		
% 		\( t_0 \)  & \( \mathrm{s} \)  & time of transition - when the surface starts warming\\	
% 		\( T \)  & \( \mathrm{K} \)  & absolute temperature\\		
% 		
% 		\( \mathbi{u} \) & \( \mathrm{m} \mathrm{s}^{-1}  \) &  vector of wind velocity  \\
% 		
% 		\( u \) & \( \mathrm{m} \mathrm{s}^{-1}  \) &  component of wind velocity in the direction of axis-\(x\)  \\
% 		
% 		\( \mathsf{U}(a,b) \)  & \( \cdot \) & uniform distribution on the interval \([a,b] \) \\
% 		\( v \) & \( \mathrm{m} \mathrm{s}^{-1}  \) &   component of wind velocity in the direction of axis-\(y\)  \\
% 		\( v_f \) & \( \mathrm{m} \mathrm{s}^{-1}  \) & large scale wind forcing in the direction of  axis-\(y\) \\		
% 		\( w \) & \( \mathrm{m} \mathrm{s}^{-1}  \) &  vertical component of wind velocity \\
% 		
% 		\( w_u \) & \( \mathrm{m} \mathrm{s}^{-1}   \) & vertical velocity in strong updraughts\\
% 		\( \overline{(w^{\prime} \varphi^{\prime})} \) & 	\( \cdot \) & vertical flux of scalar quantity; general notation \\
% 		\( \overline{(w^{\prime} \varphi^{\prime})}_{s} \) & 	\( \cdot \) & vertical flux of scalar quantity at the surface; general notation \\			
% 
% 		\( \overline{(w^{\prime} \theta^{\prime})} \) & 	\( \cdot \) 	& vertical kinematic heat flux \\ % ? adjust name? 
% 		\( \overline{(w^{\prime} \theta^{\prime}_{v})} \) & 	\( \cdot \) 	& vertical kinematic buoyancy flux \\ % ? adjust name? 
% 		\( \overline{(w^{\prime} q^{\prime})} \) & 	\( \cdot \) 		& vertical kinematic moisture flux\\ % ? adjust name ?	
% 		
% 		\( z_{0,\mathrm(vec)} \) 	& \( \mathrm{m}  \) & aerodynamic roughness length\\
% 		\( z_{0,\mathrm(vec)} \) 	& \( \mathrm{m}  \) & aerodynamic roughness length for wind \\
% 		\( z_{0,\theta} \) 		& \( \mathrm{m}  \) & aerodynamic roughness length for scalar quantities\\
% 		\( z_i \) & \( \mathrm{m}  \) & height of the mixed boundary layer (MBL) \\	
% 		\( \mathsf{z}_{\alpha} \) & \scriptsize{1}  &   quantile of the standard normal distribution for probability \(\alpha\)\\
% 		
% 		\( \delta_{i,j} \) &  \scriptsize{1} & Kronecker delta\\
% 		
% 		%-> change this one?
% 		\( \delta_{\mathrm{(h)}} T \) & \( \mathrm{K} \) & temperature scale of a heterogeneity in surface potential temperature\\
% 		\( \delta t \) & \( \mathrm{s} \) & length of timestep in a timeseries \\
% 		\( \Delta t \) & \( \mathrm{s} \) & length of a  timestep of numerical computations \\
% 		\( \Delta x \) & \( \mathrm{m} \) & grid resolution in the direction of x-axis \\
% 		\( \Delta z \) & \( \mathrm{m} \) & grid resolution in the vertical direction \\		
% 		\( \Delta_{\mathrm{(h)}} T \) & \( \mathrm{K} \) & temperature scale of a surface anomaly \\
% 		%-> check this one		
% 		\( \varepsilon_{i,j,k} \) &  \scriptsize{1} & Levi-Civita symbol in three dimensions\\
% 		\( \theta \)  & \( \mathrm{K} \)  & potential temperature \\
% 		\( \theta_{e} \)  & \( \mathrm{K} \)  & equivalent potential temperature \\		
% 		% \( \theta_{v} \)  & \( \mathrm{K} \)  & virtual potential temperature \\
% 		% \( \theta \)  & \( \mathrm{K} \)  & potential temperature \\
% 		\( \theta_{\mathrm{surf}} \) & \( \mathrm{K} \) & surface potential temperature\\	
% 		\( \theta_{v} \)  & \( \mathrm{K} \)  & virtual potential temperature \\
% 	      \\ \hline
% 	\end{tabular}
%       \end{onehalfspace}

%       \begin{onehalfspace}
%       \begin{tabular}{l l p{0.7\textwidth} }
% 	    \hline
% 	    	notation  & unit & meaning \\
% 	    \hline \\
% 		\( \kappa \) & \scriptsize{1} & Von Kármán constant \\
% 
% 
% 		\( \lambda \) & \( \mathrm{m} \) & mixing length \\
% 		\( \lambda_0 \) & \( \mathrm{m} \) & reference mixing length \\		
% 		\( \nu_m \) & \( \mathrm{m}^{2} \mathrm{s}^{-1}\)  & sub-filter eddy-viscosity in a subgrid model \\
% 		\( \nu_{m,s}\) & \( \mathrm{m}^{2} \mathrm{s}^{-1}\)  & sub-filter eddy-viscosity in the surface exchange model \\
% 		\( \nu_h \) & \( \mathrm{m}^{2} \mathrm{s}^{-1}\)  & sub-filter eddy-diffusivity in a subgrid model \\
% 		\( \nu_{h,s} \) & \( \mathrm{m}^{2} \mathrm{s}^{-1}\)  & sub-filter eddy-diffusivity in a surface exchange model\\
% 		
% 		\( \sigma_{\varphi} \)  & \( \cdot \) & standard deviation of a quantity \( \varphi \) \\	
% 		
% 		\( \varphi \) & 	\( \cdot  \) & scalar quantity; general notation \\
% 		\( \rho \)  & \( \mathrm{kg} \; \mathrm{m}^{-3} \) & density of air \\	
% 		\( \tau \)  & \( \mathrm{N} \, \mathrm{m}^{-2} \)  & vertical momentum flux; wind stress\\
% 		\( \tilde{\tau}_{i,j} \)  & \( \mathrm{N} \, \mathrm{m}^{-2} \)  & tensor of the subgrid stress\\
% 		\( \phi_m \) &  \scriptsize{1} & Businger--Dyer function for the momentum \\
% 		\( \phi_h \) &  \scriptsize{1} & Businger--Dyer function for the heat flux \\
% 	
% 	
% 		% add a note about vertical fluxes - always upward
% 		
% \\ \hline
% 	\end{tabular}
% \end{onehalfspace}

%\newpage
\section*{List of Abbreviations}
\addcontentsline{toc}{section}{List of Abbreviations}

\begin{onehalfspace}
	\begin{tabular}{l p{0.7\linewidth}}
\hline
		notation & meaning \\
\hline \\
		% AtTW 	&   after the tranisition to warm surface conditions \\  %-> add this one
		AWS	& automatic weather station \\
		%d->? or automated   - NO		
		ABL 	& atmospheric boundary layer \\
		CAO 	& cold--air outbreak \\
		CBL 	& convective boundary layer \\
% 		CFL	& Courant-Friedrichs-Lewy condition \\
% 		Cu      & cumulus \\
% 		CuL 	& cumulus layer \\
% 		EDMF	& eddy-diffusivity mass-flux \\
% 		IBL	& internal boundary layer \\
% 		IFS	& ECMWF Integrated Forecasting System \\
% 		%d -> check whether it is correct IQR : corrected
% 		IQR	& interquartile range \\
% 		EZ 	& entrainment zone \\
% 		FA 	& free atmosphere \\
% 		LES	& large eddy simulation \\
% 		LEM	& Met Office Large Eddy Model \\
% 		LH	& latent heat \\
% 		LW	& long-wave infrared radiation \\
% 		MetUM	& Met Office Unified Model \\	
% 		MIZ	& marginal sea-ice zone \\
% 		ML 	& (well-) mixed layer \\
% 		P--W	&  Piascek--Williams \\
% 	        %-> keep this one
% 		CtS 	& control set \\
		SBL 	& stable boundary layer \\
		Sc 	& stratocumulus\\
		SH	& sensible heat \\
		TKE	& turbulent kinetic energy \\
\hline
	\end{tabular}
\end{onehalfspace}

