% Tikz Thesis structure diagram
%
  \begin{tikzpicture}[node distance = 4mm, auto] 
    \node [block] (intro)  {1. Introduction};
    \node [block] (theory) at +(8,-0.5) {2. Theory of LEM};
    \node [block_special] (methods) at +(8,-2.5) {3. General Methodology};
    \node [block_special] (idealised) at +(6.5,-4.5) {4. Idealised CAO};
    \node [block_special] (adjusted) at +(6.5,-6.5) {5. Idealised~CAO Adjusted Scenarios};
    \node [block_special] (case_studies) at +(11.5,-7) {6. Case Studies of Cold-Air Outbreaks};  
    \node [block_done] (conclusions) at +(0,-8.5) {7. Conclusions};
    \draw[->,line width=0.8pt] (intro) -- (methods);
    \draw[->,line width=0.8pt] (theory) -- (methods);
    \draw[->,line width=0.8pt] (methods) -- (idealised);
 
    \draw[->,line width=0.8pt] (methods) -- (case_studies);
    \draw[->,line width=0.8pt] (idealised) -- (adjusted);
    \draw[->,line width=0.8pt] (idealised) -- (conclusions);
    \draw[->,line width=0.8pt, dashed](intro) -- (conclusions);
    \draw[-,line width=0.8pt ] (adjusted) -- (6.5, -7.5);
    \draw[-,line width=0.8pt ] (case_studies) -- (11.5, -8.5);
    \draw[->,line width=0.8pt ] (6.5, -7.5) -- (conclusions);
    \draw[->,line width=0.8pt ] (11.5, -8.5) -- (conclusions);
   \end{tikzpicture}~\hspace{-2ex}
