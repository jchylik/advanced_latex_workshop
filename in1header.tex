% ==== Header for document with math =========

% --- input -----------------------------------
  \usepackage[utf8]{inputenc}

% --- math -----------------------------------
  \usepackage{amsmath}
  \usepackage{amsfonts}
  \usepackage{amstext}
  \usepackage{amssymb}
  \usepackage{amsthm}
  

% --- graphic and formating ------------------
  % bibliography
  \usepackage[sort,comma]{natbib}
  % settings
  \usepackage{enumitem}
  \usepackage{setspace}
  \usepackage{pdflscape}
  \usepackage{graphicx}
  \usepackage{wrapfig}
  \usepackage[hypcap]{caption}
  \usepackage{subcaption}
  % \usepackage[cm]{fullpage}
  \usepackage[top=2cm, bottom=1.7cm, left=3.6cm, right=1.1cm]{geometry}
  % \usepackage[leftmargin=2.5cm]{geometry}
  %\usepackage{subfigure}
  %\usepackage{caption}
  %\usepackage{subcaption}  
  \usepackage{placeins}
  \usepackage{makeidx} 
  \usepackage{epstopdf}
  % \usepackage{tocloft}     % custom lists
  % \usepackage{minitoc}     % table of content in a chapter
  
   \usepackage{listings}     % code listings
  % \usepackage[printwatermark]{xwatermark} % \usepackage{draftwatermark}
  
  \usepackage[colorlinks=true,linkcolor=interlink,citecolor=DarkCite]{hyperref}
  
  % \usepackage{picture}
  \usepackage[usenames,dvipsnames]{color}
  \usepackage{colortbl} 
  

   \setlength{\headsep}{16pt}
%    
   \usepackage{fancyhdr}
 % ---- fancy page setting -------------- 
   %\input{style_header_footer.tex}
   \fancyhead[l]{\color{gray}{19. and 21. June 2023}}
\fancyhead[r]{\color{gray}{Informal LaTeX workshop}} % {Exam METABL ~---~ August 1, 2018}
\fancyfoot[l]{\color{gray}{intermediate and advanced topics}}
\fancyfoot[c]{\color{gray}{- \thepage/\pageref*{LastPage} -}}
\fancyfoot[r]{\color{gray}{\( \star \)}} 
\setlength{\headheight}{18pt}
    \renewcommand{\headrulewidth}{1pt}
    \renewcommand{\footrulewidth}{1pt}

  %-----------------------------
  
  % picture libraries
  \usepackage{tikz} % Required for flow chart
   \usetikzlibrary{arrows,positioning} % Tikz libraries required for the flow chart in the template
   
       \tikzset{
        %Define standard arrow tip
        >=stealth',
        %Define style for boxes
        point/.style={
           rectangle,
           rounded corners,
           draw=black, very thick,
           text width=6.5em,
           minimum height=2em,
           text centered},
        % Define arrow style
        pil/.style={
           ->,
           thick,
           shorten <=2pt,
           shorten >=2pt,}
    }

   \usepackage{csvsimple}  % csv table
   
   \usepackage{lipsum}    % filler text
   
   
  \hypersetup{
    % bookmarks=false,         % show bookmarks bar?https://www.overleaf.com/project/637266f5927ea984e19f515e
    % unicode=false,          % non-Latin characters in Acrobat’s bookmarks
    % pdftoolbar=false,        % show Acrobat’s toolbar?
    % pdfmenubar=false,        % show Acrobat’s menu?
%     % pdffitwindow=false,     % window fit to page when opened
    % pdfstartview={FitH},    % fits the width of the page to the window
    pdfinfo={
        Title={Informal LaTeX workshop},
        Author={InScAPE group},
        Creator={Your Name Here},
        Producer={Institute for Geophysics and Meteorology},
        Subject={Informal LaTeX workshop on intermediate and advanced topics},
        Keywords={tables, tikz, counters}
    },
    %pdfauthor={Author},     % author
    %pdfsubject={Subject},   % subject of the document
    %pdfcreator={Creator},   % creator of the document
    %pdfproducer={Producer}, % producer of the document
    %pdfkeywords={keyword1, key2, key3}, % list of keywords
    % pdfnewwindow=true,      % links in new PDF window
    colorlinks=true,       % false: boxed links; true: colored links
    linkcolor= ref_out,          % color of internal links (change box color with linkbordercolor)
    linkbordercolor = red, %
    urlbordercolor = {0 0.6 1},   % 
    citecolor=green,        % color of links to bibliography
    filecolor=cyan,         % color of file links
    urlcolor=magenta       % color of external links
    % pdfborderstyle={/S/U/W 1}% border style will be underline of width 1pt
} 

% ---- new commands --------------------------
  % ----- link colours ------------
    \newcommand{\reffig}[1]%
    {\hypersetup{linkcolor=figlink}%
    \ref{#1}%
    \hypersetup{linkcolor=interlink}}
 
  % ---- math symbols  ----------------------- 
  \newcommand{\e}{\mathrm{e}}
  \newcommand{\dx}{\mathrm{d}}
  \newcommand{\normal}{\mathbi{n}}
  \newcommand{\bsigma}{\mathbf{t}}
  \newcommand{\vnull}{\mathbf{0} \!\!\! ^{ _{ _{\scriptscriptstyle -}}}}
  \newcommand{\vnullt}{\mathbf{0}^{ \mathrm{T} \!\!\!\!\!\!\! \! _{ _{\scriptscriptstyle -}}}}
  % ---- math fonts -------------------------- 
  \newcommand{\mathbs}[1]{\textsf{#1}}
  \newcommand{\mathbi}[1]{\textbf{\emph #1}}
  \newcommand{\mathff}{\textbf{\textit f}}
  \newcommand{\mathbis}[1]{\textsf{\em #1}}
  \newcommand{\mathcb}[1]{\boldsymbol{\mathcal #1}}
  
  
\newcommand{\thv}{\theta_{\scriptscriptstyle \mathcal{V} }}
\newcommand{\thml}{\theta^{\scriptscriptstyle \mathrm{(ML)}}}

  % --- modifying existing commands -------------------
  

  
  % ---- our own Macros  --------------------------
  
  %  our own lists 
%  \newcommand{\listoflists}{List of Lists}  % we add a list
  
%  \newlistof{lists}{tol}{\listoflists}      % list itself 
  
%  \newcommand{\list}[1]{%
%    \refstepcounter{lists}
%    \par\noindent\textbf{lists \theexample. #1}
%    \addcontentsline{tol}{list}
%    {\protect\numberline{\thechapter.\theexample}#1}\par
  
  

% ---- latex commands -----------------------
  % --- graphic commands --------------------
  \DeclareGraphicsExtensions{.png,.jpg}

  % --- colour definition ------------------
  % colour definition
  \definecolor{GreenDone}{rgb}{0.2,0.7,0.2}
  \definecolor{lightorange}{rgb}{0.9,0.4,0}
  \definecolor{lightestorange}{rgb}{1,0.8,0.5}
  \definecolor{darkorange}{rgb}{0.2,0.1,0}
  \definecolor{interlink}{rgb}{0.6,0,0}
  \definecolor{DarkCite}{rgb}{0,0.3,0}
  % \definecolor{ref_out}{rgb}{0.3,0.3,0}
  \definecolor{ref_out}{rgb}{0.6,0,0}
  \definecolor{lightyellow}{rgb}{0.99,0.99,0.4}
  \definecolor{UeaBlue}{RGB}{0,76,103}
  \definecolor{LightBlue}{RGB}{20,20,250}
  \definecolor{DarkBlue}{RGB}{15,10,100}  
  \definecolor{figlink}{RGB}{45,10,120} 
  \definecolor{bordercol}{RGB}{40,40,40}
  \definecolor{headercol1}{RGB}{186,215,230}
  \definecolor{headercol2}{RGB}{0,76,103}
  \definecolor{headerfontcol}{RGB}{0,0,0}
  \definecolor{boxcolor}{RGB}{186,215,230}
